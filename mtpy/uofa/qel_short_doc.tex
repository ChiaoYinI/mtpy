\documentclass[11pt,DIV=12]{scrartcl}

\usepackage{mdwlist}


\begin{document}
{\centering \bf \Large QEL data processing - a short guide \par}

\section*{Scripts}

\begin{itemize}
\item C:

\begin{itemize*}
\item InterpAdl
\item Pak2Bin
\item Bin2Adl
\item Pak2AscDeci650
\end{itemize*}


\item  Python:

\begin{itemize*}
\item qel\_birrp1Hzdata2overview\_fd.py
\item qel\_birrp\_all\_days\_one\_station\_loop.py
\item qel\_birrp\_one\_day\_all\_stations\_loop.py
\item qel\_birrpinputs2mseed.py
\item qel\_concatenate\_650downsampled\_bits.py
\item qel\_convert\_1station\_timeseries.py
\item qel\_convert\_allstations.py
\item qel\_monitoring\_j2edi.py
\item qel\_rawdata2birrpinput.py
\item qel\_unpack\_and\_decimate\_1Hz.py
\item qel\_prepare\_birrp\_data3.py
\item qel\_unpack6hrs\_loop.py
\end{itemize*} 
\item  Other files:
\begin{itemize*}
\item qel\_instrument\_response\_freq\_re\_im.txt
\item qel\_short\_doc.tex
\end{itemize*}

\end{itemize}

\section{Steps}


\subsection{unpack N hour block data}

Preparation:\\
In the file \texttt{qel\_rawdata2birrpinput.py}, set the sampling rate and the length of the blocks. The length is given in hours (integer) and must be $>6$ 

Additionally you must specify the location of the compiled C-codes (see above)\\

Call:\\

\texttt{python qel\_rawdata2birrpinput.py <data directory sta A> <data directory sta B> <data directory sta C>   <channelsA> <channelsB> <channelsC>  <start time yymmdd-HHhMM> [<destination folder>] }\\
        
Example:\\

\texttt{python qel\_rawdata2birrpinput.py L206 L107 MagRR n,e,s,w n,e n,e 140318-07h15}


\subsection{1Hz data overview}

Preparation - in the header of \texttt{qel\_unpack\_and\_decimate\_1Hz.py}:

\begin{itemize*}
\item specify the executable \texttt{Pak2AscDeci650}
\item if necessary, define the temporary output filename 
\item set the base for the input data
\item set the output data folder name
\item define a 'profile\_prefix' for the stations
\item define the starting station and the number of stations
\end{itemize*}
Note: at the moment, indices go over two-digit numbers, i.e. $00 - 99$.

Call:\\

\texttt{python qel\_unpack\_and\_decimate\_1Hz.py}


\subsection{unpack 6 hour block data (legacy)}

Preparation:\\
In file \texttt{qel\_prepare\_birrp\_data3.py}
\begin{itemize*}
\item set location of SymRes (C-codes) executables
\end{itemize*}

In file \texttt{qel\_unpack6hrs\_loop.py}
\begin{itemize*}
\item set list of station name(s)
\item define list of subfolders
\item set list of dates (hard-coded so far)
\item set channels of interest
\item define output data base directory
\item set B-field data locations (and channels)
\item set prefix (according to definition of station names before)
\item define subfolder add\_on (everything that is NOT prefix+station in the naming)
\end{itemize*}

Call:\\

\texttt{python qel\_unpack6hrs\_loop.py}

\subsection{1Hz data overview (alternative)}

Preparation: use file \texttt{qel\_rawdata2birrpinput.py}, set the samplingrate to 1 and the hours to 'as long as possible'. This produces then 4 files per station for the given time. Note that they all start at the same time. 


Call:\\

\texttt{python qel\_rawdata2birrpinput.py <data directory sta A> <data directory sta B> <data directory sta C>  n,e,s,w n,e,s,w n,e,s,w   <start time> <destination folder> }\\

The outputs can be concatenated into a 4-column first differences file using \\

\texttt{python qel\_birrp1Hzdata2overview\_fd.py <input dir> <output dir> <stationname A> <stationname B> <stationname C> <prefix>}

( 'prefix' is everything predeeding the \texttt{'.timestamp'}, but WITHOUT the \texttt{staX} bit)


\subsection{BIRRP loop -- fixed station, variable days}

Preparation:

In file \texttt{qel\_birrp\_all\_days\_one\_station\_loop.py}
\begin{itemize*}
\item set input base directory name 
\item set output directory name 
\item define station name
\item define BIRRP executable
\item set list of frequencies to be notch-filtered
\item define channels to be used for E-field data
\end{itemize*}

The actual BIRRP parameters are hardcoded in a string in the lower header section. A good set of parameters should be established first (without this loop script), and then edited. Afterwards, it should not be necessary to change them during processing.\\


The input data directory must contain subdirectories, sorted and named by days.


The name of the subdirectories must end with a date, separated from the rest
of the name by an underscore.

date format: YYMMDD (e.g. 140320 for the 20th of March 2014) 


Call:\\

\texttt{python qel\_birrp\_all\_days\_one\_station\_loop.py}


\subsection{BIRRP loop -- fixed day, variable stations}

Preparation:

In file \texttt{qel\_birrp\_all\_days\_one\_station\_loop.py}
\begin{itemize*}
\item set input base directory name 
\item set output directory name 
\item define date 
\item define BIRRP executable
\item set list of frequencies to be notch-filtered
\item define channels to be used for E-field data
\end{itemize*}
The input data directory must contain subdirectories, sorted and named by stations.

The name of the subdirectories must start with a station name. If the name is 
longer, it must be  separated from the rest of the name by an underscore.


Call:\\

\texttt{python qel\_birrp\_all\_days\_one\_station\_loop.py }


\subsection{convert BIRRP output -- fixed station, variable days }

Preparation:


In the file \texttt{ qel\_convert\_1station\_timeseries.py}
\begin{itemize*}
\item set input base directory name 
\item set output directory name
\item define station name
\item define plot\_component\_dict (can be empty, but must be present)
\item define survey config file location
\item define calibration file location
\item (legacy) optional: define a list of strings, which shall be stripped from filenames during the handling 
\end{itemize*}

Convert the outputs of BIRRP. Loop over all dayfolders/blocks for a single 
station. The directory structure must be:
- input dir (to be set in the header of this file)
 - dayfolders (naming: YYMMDD)
     - outputfiles (only one of each file type *.2c2 and *.j will be processed)

Output:
- new directory (to be set in the header of this file)
    - stationname (to be set in the header of this file)
        - subdirectories 'coh columns edi plots'


Requires: 
survey\_configfile, 
calibrationfile,
plot component dictionary (can be empty)
 
 all to be set in the header of this file


The EDI file prefix is set in  "qel\_monitoring\_j2edi.py"


Call:\\

\texttt{python qel\_convert\_1station\_timeseries.py}



\subsection{convert BIRRP output -- fixed day, variable stations}

Preparation:\\

In the file \texttt{qel\_convert\_allstations.py}
\begin{itemize*}
\item set input base directory name 
\item set output directory name
\item define station name
\item define plot\_component\_dict (can be empty, but must be present)
\item define survey config file location
\item define calibration file location
\item (legacy) optional: define a list of strings, which shall be stripped from filenames during the handling 
\end{itemize*}
Convert the outputs of BIRRP. Loop over all stations for a single day.
The directory structure must be:
- input dir (to be set in the header of this file)
 - stationname (at least starting with it and then separated by underscore)
     -date (at least ending with it and  separated by underscore)
       - outputfiles (only one of each file type *.2c2 and *.j will be processed)

Output:
- new directory (to be set in the header of this file)
    - date (to be set in the header of this file)
        - subdirectories 'coh columns edi plots'


Requires: 
survey\_configfile, 
calibrationfile,
plot component dictionary (can be empty)
 
 all to be set in the header of this file


Call:\\

\texttt{python qel\_convert\_allstations.py}






\subsection{convert BIRRP input to MiniSEED}


Preparation:

Have the BIRRP input files ready in one folder. This script loops automatically over all available channels. The timing and sampling is read from the \texttt{*.timestamps} file in the respective folder.\\

Call:\\

\texttt{python qel\_birrpinputs2mseed.py <input dir> <output dir><stationname> <prefix>}\\
('prefix' is everything preceding the '\texttt{.timestamps}', incl. \texttt{staX} )


\section*{Comments}
\begin{itemize*}
\item user \textit{qel} available on Feynman and Gauss - password is \textit{'password'}
\item Gauss Python is old (version 2.4) - this causes multiple issues (formatting and missing modules/attributes). Try Feynman or personal desktops instead!
\item BIRRP runs best on Gauss
\item always use \textit{screen} to  work in emulated detachable terminals
\end{itemize*}


\end{document}