\section{\textit{combineedis} (Script) }
\label{sec:utils.combineedis}

combine two edi files into one a number of files

\section{\textit{csvutm} (Script)}
\label{sec:utils.csvutm}

Convert and add columns for different coordinate systems to a CSV file.

This script requires pyproj installed. If you have a CSV file with two columns
containing x and y coordinates in some coordinate system (the "from" system),
this script will add another two columns with the coordinates transformed into
another system (the "to" system). The CSV file must have a header row, and
the column names should be specified on the command line as well.

\subsection{\textit{csvutm} (Function)}
\label{ssec:utils.csvutm.csvutm}

...

\subsection{\textit{get\_parser} (Function)}
\label{ssec:utils.csvutm.get_parser}


...


%======================================================


\section{\textit{edidms2deg} (Script)}
\label{sec:utils.edidms2deg}

??????????

\subsection{\textit{dms2deg} (Function)}
\label{ssec:utils.edidms2deg.dms2deg}

...

\subsection{\textit{deg2dms} (Function)}
\label{ssec:utils.edidms2deg.deg2dms}

...

\subsection{\textit{check\_format} (Function)}
\label{ssec:utils.edidms2deg.check_format}

...

\subsection{\textit{edidms2deg} (Function)}
\label{ssec:utils.edidms2deg.edidms2deg}


converts the LAT/LON information within given EDI files from/to the degree/deg:min:sec format.

    Call it with (a list of) files as arguments. Wildcards are
    allowed. The repective lines within the files are converted and
    the files written to the current working directory. The names are
    not changed, so existing files get overwritten!!


%======================================================




\section{\textit{latlongutmconversion} (Script)}
\label{sec:utils.latlongutmconversion}

 Lat Long - UTM, UTM - Lat Long conversions


\subsection{\textit{LLtoUTM} (Function)}
\label{ssec:utils.latlongutmconversion.LLtoUTM}

converts lat/long to UTM coords.  Equations from USGS Bulletin 1532 
    East Longitudes are positive, West longitudes are negative. 
    North latitudes are positive, South latitudes are negative
    Lat and Long are in decimal degrees
    Written by Chuck Gantz- chuck.gantz@globalstar.com  

\subsection{\textit{UTMtoLL} (Function)}
\label{ssec:utils.latlongutmconversion.UTMtoLL}

converts UTM coords to lat/long.  Equations from USGS Bulletin 1532 
    East Longitudes are positive, West longitudes are negative. 
    North latitudes are positive, South latitudes are negative
    Lat and Long are in decimal degrees. 
    Written by Chuck Gantz- chuck.gantz@globalstar.com
    Converted to Python by Russ Nelson <nelson@crynwr.com>


%======================================================

\section{\textit{modem2vtk3d} }
\label{sec:utils.modem2vtk3d}


Build VTK files for visualising ModEM3D modelfile and station locations

Orientation convention:

coordinate system NED is used! First component is positive from South to North,
 second component is positve from West to East, third component is positive Downwards


\subsection{\textit{model2vtkgrid} (Function)}
\label{ssec:utils.modem2vtk3d.model2vtkgrid}

    Convert ModEM output files (model and responses) into 3D VTK resistivity grid



\subsection{\textit{data2vtkstationsgrid} (Function)}
\label{ssec:utils.modem2vtk3d.data2vtkstationsgrid}


    Convert ModEM data file into 2D VTK station set (unstructured grid)



\section{\textit{pakasc2TSasc} (Script)}
\label{sec:utils.pakasc2TSasc}

Convert the generic ascii tables generated by Pak2Asc (from e-logger data) into 5-column data asciis 
(time in epochs and 4 channels)


\subsection{\textit{pakascii2TSascii} (Function)}
\label{ssec:utils.pakasc2TSasc.pakascii2TSascii}

the conversion function, called by a wrapper


\section{\textit{runparalanamt} (Script) }
\label{sec:utils.runparalanamt}


process a single station using information from files. 

If you use the optimized birrp.exe and you get an error saying: didn't find all
the .bf files, something a miss.  Try running using the old version.  Something
in the optimization compiler changes something in the variable useage in the
Fortran program.  

There is a beep when it finishes and can be loud depending on the computer
volume.

At the end I've added a plot section, if you want to save the plots change
the n to y, and hopefully that will work, not sure about windows 7.  If that 
doesn't work you can leave it at n and save the plot when it comes up by
clicking on the save icon on the plot and save manually.

It the rotation angles are not correct or you want to change birrp parameters
change it in the processinginfofile and rerun the program.

Good luck

Jared Peacock 2011

\section{\textit{ws2para} (Script)}
\label{sec:utils.ws2para}

Convert ws3Dinv output files (model and responses) into 3D VTK resistivity grid
    and unstructured VTKGrid containing station locations.
