\section{\textit{birrptools} }
\label{sec:processing.birrptools}

Tools for dealing with the (Fortran) processing software BIRRP 

\subsection{\textit{finishBeep} (Function)}
\label{ssec:processing.birrptools.finishBeep}

\subsection{\textit{readProDict} (Function)}
\label{ssec:processing.birrptools.readProDict}

\subsection{\textit{scriptfilePrep} (Function)}
\label{ssec:processing.birrptools.scriptfilePrep}

\subsection{\textit{writeScriptfile} (Function)}
\label{ssec:processing.birrptools.writeScriptfile}

\subsection{\textit{callBIRRP} (Function)}
\label{ssec:processing.birrptools.callBIRRP}

\subsection{\textit{convertBIRRPoutputs} (Function)}
\label{ssec:processing.birrptools.convertBIRRPoutputs}

\subsection{\textit{plotBFfiles} (Function)}
\label{ssec:processing.birrptools.plotBFfiles}

\subsection{\textit{writeLogfile} (Function)}
\label{ssec:processing.birrptools.writeLogfile}

\subsection{\textit{runBIRRPpp} (Function)}
\label{ssec:processing.birrptools.runBIRRPpp}

\subsection{\textit{sigfigs} (Function)}
\label{ssec:processing.birrptools.sigfigs}

\subsection{\textit{sil} (Function)}
\label{ssec:processing.birrptools.sil}

\subsection{\textit{read2c2} (Function)}
\label{ssec:processing.birrptools.read2c2}



\subsection{\textit{writecoh} (Function)}
\label{ssec:processing.birrptools.writecoh}

\subsection{\textit{readcoh} (Function)}
\label{ssec:processing.birrptools.readcoh}



\subsection{\textit{bbcalfunc} (Function)}
\label{ssec:processing.birrptools.readj}

\subsection{\textit{readrf} (Function)}
\label{ssec:processing.birrptools.readrf}

\subsection{\textit{bbconvz} (Function)}
\label{ssec:processing.birrptools.bbconvz}

\subsection{\textit{lpconvz} (Function)}
\label{ssec:processing.birrptools.lpconvz}

\subsection{\textit{writeimp} (Function)}
\label{ssec:processing.birrptools.writeimp}

\subsection{\textit{readimp} (Function)}
\label{ssec:processing.birrptools.readimp}

\subsection{\textit{writedat} (Function)}
\label{ssec:processing.birrptools.writedat}

\subsection{\textit{readdat} (Function)}
\label{ssec:processing.birrptools.readdat}

\subsection{\textit{writeedi} (Function)}
\label{ssec:processing.birrptools.writeedi}

\subsection{\textit{readini} (Function)}
\label{ssec:processing.birrptools.readini}

\subsection{\textit{writeini} (Function)}
\label{ssec:processing.birrptools.writeini}


%===============================================================


\section{\textit{runbirrpsinglestation} (Script) }
\label{sec:processing.runbirrpsinglestation}

Program to process a single station using information from files. 

If you use the optimized birrp.exe and you get an error saying: didn't find all
the .bf files, something a miss.  Try running using the old version.  Something
in the optimization compiler changes something in the variable useage in the
Fortran program.  

At the end I've added a plot section, if you want to save the plots change
the n to y, and hopefully that will work, not sure about windows 7.  If that 
doesn't work you can leave it at n and save the plot when it comes up by
clicking on the save icon on the plot and save manually.

It the rotation angles are not correct or you want to change birrp parameters
change it in the processinginfofile and rerun the program.

%===============================================================



\section{\textit{striketools} }
\label{sec:processing.striketools}

Tools for STRIKE software (?) McNeice, G. W.and Jones, A. G., 2001, Multisite, multifrequency tensor 
            decomposition of magnetotelluric data: *Geophysics*, **66**, 
            pg. 158--173.

All functionality is within one class

\subsection{\textit{Strike} (Class)}
\label{ssec:processing.striketools.Strike}

Strike class ... ???

\subsubsection{\textit{readDCMP} (Method)}
\label{sssec:processing.striketools.Strike.readDCMP}

read in the output file .dcmp

\subsubsection{\textit{plotHistogram} (Method)}
\label{sssec:processing.striketools.Strike.plotHistogram}

plot the histogram of the different angles

\subsubsection{\textit{plotAngles} (Method)}
\label{sssec:processing.striketools.Strike.plotAngles}

plot the angles vs. log(period)

\subsubsection{\textit{plotResPhase} (Method)}
\label{sssec:processing.striketools.Strike.plotResPhase}

plot the regional resistivity and phase


%===============================================================


\section{\textit{tftools} }
\label{sec:processing.tftools}

Collection of functions for Time-Frequency-Analysis of MT (raw-)data

\subsection{\textit{padzeros} (Function)}
\label{ssec:processing.tftools.padzeros}

return a function that is padded with zeros to the next
    power of 2

\subsection{\textit{sfilter} (Function)}
\label{ssec:processing.tftools.sfilter}

apply a sinc filter of width w to the function f by multipling in
    the frequency domain

\subsection{\textit{dctrend} (Function)}
\label{ssec:processing.tftools.dctrend}

remove a dc trend from the function 

\subsection{\textit{normalizeL2} (Function)}
\label{ssec:processing.tftools.normalizeL2}

normalizeL2(f) will return the function f normalized by the L2 norm 

\subsection{\textit{decimatef} (Function)}
\label{ssec:processing.tftools.decimatef}

Will decimate a function by the factor m. First an 8th order Cheybechev 
    type I filter with a cuttoff frequency of .8/m  is applied in both 
    directions to minimize any phase distortion and remove any aliasing. Note 
    decimation values above 10 will typically result in bad coefficients, 
    therefore if you decimation is more than 10 just repeat the decimation until
    the desired decimation is reached.

\subsection{\textit{dwindow} (Function)}
\label{ssec:processing.tftools.dwindow}

Calculates the derivative of the given window

\subsection{\textit{gausswin} (Function)}
\label{ssec:processing.tftools.gausswin}

gausswin will compute a gaussian window of length winlen with a variance

\subsection{\textit{wvdas} (Function)}
\label{ssec:processing.tftools.wvdas}

compute the analytic signal for WVVD as defined by \
    J. M. O' Toole, M. Mesbah, and B. Boashash, (2008), "A New Discrete Analytic\
    Signal for Reducing Aliasing in the Discrete Wigner-Ville Distribution", \
    IEEE Trans.  on Signal Processing,

\subsection{\textit{stft} (Function)}
\label{ssec:processing.tftools.stft}

calculate the spectrogam of
    the given function by calculating the fft of a window of length nh at each
    time instance with an interval of tstep.  The frequency resolution is nfbins
    Can compute the cross STFT by inputting fx as [fx1,fx2]

\subsection{\textit{reassignedstft} (Function)}
\label{ssec:processing.tftools.reassignedstft}

compute the reassigned spectrogram by estimating the center of gravity of 
    the signal and condensing dispersed energy back to that location.
   

\subsection{\textit{wvd} (Function)}
\label{ssec:processing.tftools.wvd}

calculate the 
    Wigner-Ville distribution for a function f. Can compute the cross spectra
    by inputting fx as [fx1,fx2] 

\subsection{\textit{spwvd} (Function)}
\label{ssec:processing.tftools.spwvd}

calculate the smoothed pseudo Wigner-Ville distribution for a function
    fx. smoothed with Gaussians windows to get best localization. 

\subsection{\textit{robustwvd} (Function)}
\label{ssec:processing.tftools.robustwvd}

calculate the smoothed pseudo Wigner-Ville distribution for a function
    fx. smoothed with Gaussians windows to get best localization. 


\subsection{\textit{specwv} (Function)}
\label{ssec:processing.tftools.specwv}

calculate 
    the Wigner-Ville distribution mulitplied by the STFT windowed by the common 
    gaussian window h for a function f. 

\subsection{\textit{modifiedb} (Function)}
\label{ssec:processing.tftools.modifiedb}

calculate the modified b distribution as defined by $\cosh(n)^-2$ beta 
    for a function fx. 

\subsection{\textit{robuststftMedian} (Function)}
\label{ssec:processing.tftools.robuststftMedian}

output an array 
    of the time-frequency robust spectrogram calculated using the vector median
    simplification.


\subsection{\textit{robuststftL} (Function)}
\label{ssec:processing.tftools.robuststftL}

output an array of the
    time-frequency robust spectrogram by estimating the vector median and 
    summing terms estimated by alpha coefficients.

\subsection{\textit{smethod} (Function)}
\label{ssec:processing.tftools.smethod}

calculate
    the smethod by estimating the STFT first and computing the WV of window 
    length L in the frequency domain.  For larger L more of WV estimation, if 
    L=0 get back STFT

\subsection{\textit{robustSmethod} (Function)}
\label{ssec:processing.tftools.robustSmethod}

computes the 
    robust Smethod via the robust spectrogram.

\subsection{\textit{reassignedSmethod} (Function)}
\label{ssec:processing.tftools.reassignedSmethod}

calulate the reassigned S-method as described by Djurovic[1999] by 
    using the spectrogram to estimate the reassignment

\subsection{\textit{plottf} (Function)}
\label{ssec:processing.tftools.plottf}

plot a calculated tfarray 

\subsection{\textit{plotAll} (Function)}
\label{ssec:processing.tftools.plotAll}

plot a calculated tfarray with 
    limits corresponding to tlst and flst.

\subsection{\textit{stfbss} (Function)}
\label{ssec:processing.tftools.stfbss}

estimates sources using a blind source algorithm based on spatial 
    time-frequency distributions.  At the moment this algorithm uses the SPWVD 
    to estimate TF distributions. 
