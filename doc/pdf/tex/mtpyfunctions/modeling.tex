\section{\textit{modemtools} }
\label{sec:modeling.modemtools}

Collection of functions for dealing with ModEM. So far there are conversions from Ws3Dinv models and data into ModEM3D model and data. Furthermore, WingLink-generated model meshes can be converted.

\subsection{\textit{winglinkmesh2modelfile} (Function)}
\label{ssec:.modeling.modemtools.winglinkmesh2modelfile}

Convert a WingLink mesh geometry into ModEM modelfile (so far only homogeneous halfspace model supported)


\subsection{\textit{latlon2xy} (Function)}
\label{ssec:.modeling.modemtools.latlon2xy}

UNUSED

\subsection{\textit{edis2datafile} (Function)}
\label{ssec:.modeling.modemtools.edis2datafile}

Generate ModEM datafile from a combination of a WingLink model .out-file and a list of EDI files

\subsection{\textit{generate\_edilist} (Function)}
\label{ssec:.modeling.modemtools.generate_edilist}

browse a folder for existing EDI files. Sensitive to file suffix 'EDI',/'edi'

\subsection{\textit{winglink2modem} (Function)}
\label{ssec:.modeling.modemtools.winglink2modem}

    Conversion of WingLink output files into ModEM input. Wrapper fro 2 functions, which deal with the model- and data-file respectively.


\subsection{\textit{wsinv2modem\_data} (Function)}
\label{ssec:.modeling.modemtools.wsinv2modem_data}

    Convert an existing input data file from Weerachai's wsinv style into Egbert's ModEM type


\subsection{\textit{wsinv2modem\_model} (Function)}
\label{ssec:.modeling.modemtools.wsinv2modem_model}

    Convert an existing input model file from Weerachai's wsinv style into Egbert's ModEM type


\subsection{\textit{plotmodel3d} (Function)}
\label{ssec:.modeling.modemtools.plotmodel3d}

ToDo !!! void function !!! 

\subsection{\textit{getmeshblockcoordinates} (Function)}
\label{ssec:.modeling.modemtools.getmeshblockcoordinates}

Read a ModEM-style model file and return a list of 3 lists, which again contain the X/Y/Z coordinate of a mesh block


%===============================================================================


\section{\textit{occamtools} }
\label{sec:modeling.occamtools}



\subsection{\textit{Occam1D} (Class)}
\label{ssec:.modeling.occamtools.Occam1D}



\subsubsection{\textit{make1DdataFile} (Method)}
\label{sssec:.modeling.occamtools.Occam1D.make1DdataFile}



\subsubsection{\textit{make1DModelFile} (Method)}
\label{sssec:.modeling.occamtools.Occam1D.make1DModelFile}



\subsubsection{\textit{make1DInputFile} (Method)}
\label{sssec:.modeling.occamtools.Occam1D.make1DInputFile}



\subsubsection{\textit{read1DModelFile} (Method)}
\label{sssec:.modeling.occamtools.Occam1D.read1DModelFile}



\subsubsection{\textit{read1DInputFile} (Method)}
\label{sssec:.modeling.occamtools.Occam1D.read1DInputFile}



\subsubsection{\textit{read1DdataFile} (Method)}
\label{sssec:.modeling.occamtools.Occam1D.read1DdataFile}



\subsubsection{\textit{read1DIterFile} (Method)}
\label{sssec:.modeling.occamtools.Occam1D.read1DIterFile}



\subsubsection{\textit{read1DRespFile} (Method)}
\label{sssec:.modeling.occamtools.Occam1D.read1DRespFile}



\subsubsection{\textit{plot1D} (Method)}
\label{sssec:.modeling.occamtools.Occam1D.plot1D}



\subsubsection{\textit{plotL2Curve} (Method)}
\label{sssec:.modeling.occamtools.Occam1D.plotL2Curve}



\subsection{\textit{getdatetime} (Function)}
\label{ssec:.modeling.occamtools.getdatetime}



\subsection{\textit{makestartfiles} (Function)}
\label{ssec:.modeling.occamtools.makestartfiles}



\subsection{\textit{writemeshfile} (Function)}
\label{ssec:.modeling.occamtools.writemeshfile}



\subsection{\textit{writemodelfile} (Function)}
\label{ssec:.modeling.occamtools.writemodelfile}



\subsection{\textit{writestartupfile} (Function)}
\label{ssec:.modeling.occamtools.writestartupfile}



\subsection{\textit{read\_datafile} (Function)}
\label{ssec:.modeling.occamtools.read_datafile}



\subsection{\textit{get\_model\_setup} (Function)}
\label{ssec:.modeling.occamtools.get_model_setup}



\subsection{\textit{blocks\_elements\_setup} (Function)}
\label{ssec:.modeling.occamtools.blocks_elements_setup}




\subsection{\textit{OccamPointPicker} (Class)}
\label{ssec:.modeling.occamtools.OccamPointPicker}



\subsubsection{\textit{inAxes} (Method)}
\label{sssec:.modeling.occamtools.OccamPointPicker.inAxes}



\subsubsection{\textit{inFigure} (Method)}
\label{sssec:.modeling.occamtools.OccamPointPicker.inFigure}



\subsubsection{\textit{on\_close} (Method)}
\label{sssec:.modeling.occamtools.OccamPointPicker.on_close}



\subsection{\textit{Occam2DData} (Class)}
\label{ssec:.modeling.occamtools.Occam2DData}




\subsubsection{\textit{make2DdataFile} (Method)}
\label{sssec:.modeling.occamtools.Occam2DData.make2DdataFile}



\subsubsection{\textit{read2DdataFile} (Method)}
\label{sssec:.modeling.occamtools.Occam2DData.read2DdataFile}



\subsubsection{\textit{rewrite2DdataFile} (Method)}
\label{sssec:.modeling.occamtools.Occam2DData.rewrite2DdataFile}



\subsubsection{\textit{plotMaskPoints} (Method)}
\label{sssec:.modeling.occamtools.Occam2DData.plotMaskPoints}



\subsubsection{\textit{maskPoints} (Method)}
\label{sssec:.modeling.occamtools.Occam2DData.maskPoints}



\subsubsection{\textit{read2DRespFile} (Method)}
\label{sssec:.modeling.occamtools.Occam2DData.read2DRespFile}



\subsubsection{\textit{plot2DResponses} (Method)}
\label{sssec:.modeling.occamtools.Occam2DData.plot2DResponses}



\subsubsection{\textit{plotPseudoSection} (Method)}
\label{sssec:.modeling.occamtools.Occam2DData.plotPseudoSection}



\subsubsection{\textit{plotAllResponses} (Method)}
\label{sssec:.modeling.occamtools.Occam2DData.plotAllResponses}





\subsection{\textit{Occam2DModel} (Class)}
\label{ssec:.modeling.occamtools.Occam2DModel}



\subsubsection{\textit{read2DIter} (Method)}
\label{sssec:.modeling.occamtools.Occam2DModel.read2DIter}



\subsubsection{\textit{read2DInmodel} (Method)}
\label{sssec:.modeling.occamtools.Occam2DModel.read2DInmodel}



\subsubsection{\textit{read2DMesh} (Method)}
\label{sssec:.modeling.occamtools.Occam2DModel.read2DMesh}



\subsubsection{\textit{get2DData} (Method)}
\label{sssec:.modeling.occamtools.Occam2DModel.get2DData}



\subsubsection{\textit{get2DModel} (Method)}
\label{sssec:.modeling.occamtools.Occam2DModel.get2DModel}



\subsubsection{\textit{plot2DModel} (Method)}
\label{sssec:.modeling.occamtools.Occam2DModel.plot2DModel}



\subsubsection{\textit{plotL2Curve} (Method)}
\label{sssec:.modeling.occamtools.Occam2DModel.plotL2Curve}



\subsubsection{\textit{plotDepthModel} (Method)}
\label{sssec:.modeling.occamtools.Occam2DModel.plotDepthModel}



\subsection{\textit{compare2DIter} (Function)}
\label{sssec:.modeling.occamtools.compare2DIter}





%===============================================================================

\section{\textit{winglinktools} }
\label{sec:modeling.winglinktools}


Collection of tools for dealing with output from Windows WingLink software

\subsection{\textit{readOutputFile} (Function)}
\label{ssec:.modeling.winglinktools.readOutputFile}

read an output file from winglink and output data
    in the form of a dictionary.

\subsection{\textit{plotResponses} (Function)}
\label{ssec:.modeling.winglinktools.plotResponses}

plot the responses modeled from winglink against the
    observed data.

\subsection{\textit{readModelFile} (Function)}
\label{ssec:.modeling.winglinktools.readModelFile}

read  the XYZ .txt-file output by Winglink.

(profile directions restricted to NS or EW !!!)

\subsection{\textit{readWLOutFile} (Function)}
\label{ssec:.modeling.winglinktools.readWLOutFile}

read .out file from winglink

\subsection{\textit{readSitesFile} (Function)}
\label{ssec:.modeling.winglinktools.readSitesFile}

read sites file output from winglink

\subsection{\textit{readSitesFile2} (Function)}
\label{ssec:.modeling.winglinktools.readSitesFile2}

read sites file output from winglink - alternative

more consistent output!

\subsection{\textit{getXY} (Function)}
\label{ssec:.modeling.winglinktools.getXY}

get x (e-w) and y (n-s) position of station and put in middle of cell

connection x-East and y-North here !!!! (ToDo change this)

\subsection{\textit{getmeshblockcoordinates} (Function)}
\label{ssec:.modeling.winglinktools.getmeshblockcoordinates}

(slightly different from ModEM mesh block coordinates!!)
return a list of 3 lists, which again contain the X/Y/Z coordinate of a mesh block

    Orientation is X-North, Y-East, Z-Down.
    Horizontal origin is in the center of the mesh,
    Indexing starts at the lower left (SouthWest) corner


%===============================================================================

\section{\textit{ws3dtools} }
\label{sec:modeling.ws3dtools}

Handling Weerachai's Ws3Dinv code


\subsection{\textit{ListPeriods} (Class)}
\label{ssec:modeling.ws3dtools.ListPeriods}

class to pick periods

\subsubsection{\textit{connect} (Method)}
\label{sssec:modeling.ws3dtools.ListPeriods.connect}

?

\subsubsection{\textit{onclick} (Method)}
\label{sssec:modeling.ws3dtools.ListPeriods.onclick}

?

\subsubsection{\textit{disconnect} (Method)}
\label{sssec:modeling.ws3dtools.ListPeriods.disconnect}

?


\subsection{\textit{writeWSDataFile} (Function)}
\label{ssec:modeling.ws3dtools.writeWSDataFile}

writes a data file for WSINV3D from winglink outputs

\subsection{\textit{writeInit3DFile} (Function)}
\label{ssec:modeling.ws3dtools.writeInit3DFile}

Makes an init3d file for WSINV3D

\subsection{\textit{writeStartupFile} (Function)}
\label{ssec:modeling.ws3dtools.writeStartupFile}

makes a startup file for WSINV3D

\subsection{\textit{readDataFile} (Function)}
\label{ssec:modeling.ws3dtools.readDataFile}

read in Ws3Dinv data file


\subsection{\textit{plotDataResPhase} (Function)}
\label{ssec:modeling.ws3dtools.plotDataResPhase}

plot responses from the data file and if there is a response file

\subsection{\textit{plotTensorMaps} (Function)}
\label{ssec:modeling.ws3dtools.plotTensorMaps}

plot phase tensor maps for data and or response, each figure is of a
    different period.  If response is input a third column is added which is 
    the residual phase tensor showing where the model is not fitting the data 
    well.  The data are plotted in km in units of ohm-m.

\subsection{\textit{readModelFile} (Function)}
\label{ssec:modeling.ws3dtools.readModelFile}

read in a Ws3Dinv model file

