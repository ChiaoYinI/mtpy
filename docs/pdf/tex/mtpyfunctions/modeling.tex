\section{\textit{modemtools} }
\label{sec:modeling.modemtools}

Collection of functions for dealing with ModEM. So far there are conversions from Ws3Dinv models and data into ModEM3D model and data. Furthermore, WingLink-generated model meshes can be converted.

\subsection{\textit{winglinkmesh2modelfile} (Function)}
\label{ssec:.modeling.modemtools.winglinkmesh2modelfile}

Convert a WingLink mesh geometry into ModEM modelfile (so far only homogeneous halfspace model supported)


\subsection{\textit{latlon2xy} (Function)}
\label{ssec:.modeling.modemtools.latlon2xy}

UNUSED

\subsection{\textit{edis2datafile} (Function)}
\label{ssec:.modeling.modemtools.edis2datafile}

Generate ModEM datafile from a combination of a WingLink model .out-file and a list of EDI files

\subsection{\textit{generate\_edilist} (Function)}
\label{ssec:.modeling.modemtools.generate_edilist}

browse a folder for existing EDI files. Sensitive to file suffix 'EDI',/'edi'

\subsection{\textit{winglink2modem} (Function)}
\label{ssec:.modeling.modemtools.winglink2modem}

    Conversion of WingLink output files into ModEM input. Wrapper fro 2 functions, which deal with the model- and data-file respectively.


\subsection{\textit{wsinv2modem\_data} (Function)}
\label{ssec:.modeling.modemtools.wsinv2modem_data}

    Convert an existing input data file from Weerachai's wsinv style into Egbert's ModEM type


\subsection{\textit{wsinv2modem\_model} (Function)}
\label{ssec:.modeling.modemtools.wsinv2modem_model}

    Convert an existing input model file from Weerachai's wsinv style into Egbert's ModEM type


\subsection{\textit{plotmodel3d} (Function)}
\label{ssec:.modeling.modemtools.plotmodel3d}

ToDo !!! void function !!! 

\subsection{\textit{getmeshblockcoordinates} (Function)}
\label{ssec:.modeling.modemtools.getmeshblockcoordinates}

Read a ModEM-style model file and return a list of 3 lists, which again contain the X/Y/Z coordinate of a mesh block


%===============================================================================


\section{\textit{occamtools} }
\label{sec:modeling.occamtools}

Tools for OCCAM handling


\subsection{\textit{Occam1D} (Class)}
\label{ssec:.modeling.occamtools.Occam1D}

    ==============================================\\
    This class will deal with everything occam 1D\\
    =============================================\\

\subsubsection{\textit{make1DdataFile} (Method)}
\label{sssec:.modeling.occamtools.Occam1D.make1DdataFile}

write a data file for Occam1D

\subsubsection{\textit{make1DModelFile} (Method)}
\label{sssec:.modeling.occamtools.Occam1D.make1DModelFile}

Makes a 1D model file for Occam1D.  

\subsubsection{\textit{make1DInputFile} (Method)}
\label{sssec:.modeling.occamtools.Occam1D.make1DInputFile}

Make a 1D input file for Occam 1D

\subsubsection{\textit{read1DModelFile} (Method)}
\label{sssec:.modeling.occamtools.Occam1D.read1DModelFile}

read in model 1D file

\subsubsection{\textit{read1DInputFile} (Method)}
\label{sssec:.modeling.occamtools.Occam1D.read1DInputFile}

reads in a 1D input file

\subsubsection{\textit{read1DdataFile} (Method)}
\label{sssec:.modeling.occamtools.Occam1D.read1DdataFile}

reads a 1D data file

\subsubsection{\textit{read1DIterFile} (Method)}
\label{sssec:.modeling.occamtools.Occam1D.read1DIterFile}

read an 1D iteration file
        

\subsubsection{\textit{read1DRespFile} (Method)}
\label{sssec:.modeling.occamtools.Occam1D.read1DRespFile}

read response file

\subsubsection{\textit{plot1D} (Method)}
\label{sssec:.modeling.occamtools.Occam1D.plot1D}

Plots the results of a 1D inversion.  The left plot is the response
        and the right hand plot is the model as a function of depth.

\subsubsection{\textit{plotL2Curve} (Method)}
\label{sssec:.modeling.occamtools.Occam1D.plotL2Curve}

Plot the L curve for RMS vs Iteration and RMS vs Roughness.

\subsection{\textit{getdatetime} (Function)}
\label{ssec:.modeling.occamtools.getdatetime}

calls current time:  \texttt{time.asctime(time.gmtime())}

\subsection{\textit{makestartfiles} (Function)}
\label{ssec:.modeling.occamtools.makestartfiles}

returns start files (MeshF,ModF,SF)


\subsection{\textit{writemeshfile} (Function)}
\label{ssec:.modeling.occamtools.writemeshfile}

.

\subsection{\textit{writemodelfile} (Function)}
\label{ssec:.modeling.occamtools.writemodelfile}

.

\subsection{\textit{writestartupfile} (Function)}
\label{ssec:.modeling.occamtools.writestartupfile}

.

\subsection{\textit{read\_datafile} (Function)}
\label{ssec:.modeling.occamtools.read_datafile}

    \#RELYING ON A CONSTANT FORMAT, ACCESSING THE PARTS BY COUNTING OF LINES!!!:


\subsection{\textit{get\_model\_setup} (Function)}
\label{ssec:.modeling.occamtools.get_model_setup}

??

\subsection{\textit{blocks\_elements\_setup} (Function)}
\label{ssec:.modeling.occamtools.blocks_elements_setup}

??


\subsection{\textit{OccamPointPicker} (Class)}
\label{ssec:.modeling.occamtools.OccamPointPicker}

This class helps the user interactively pick points to mask and add 
    error bars. 
    
    Usage:\\
    -------\\
    To mask just a single point right click over the point and a gray point 
    will appear indicating it has been masked
    
    To mask both the apparent resistivity and phase left click over the point.
    Gray points will appear over both the apparent resistivity and phase.  
    Sometimes the points don't exactly matchup, haven't quite worked that bug
    out yet, but not to worry it picks out the correct points
    
    To add error bars to a point click the middle or scroll bar button.  This
    only adds error bars to the point and does not reduce them so start out
    with reasonable errorbars.  You can change the increment that the error
    bars are increased with reserrinc and phaseerrinc.


\subsubsection{\textit{inAxes} (Method)}
\label{sssec:.modeling.occamtools.OccamPointPicker.inAxes}

gets the axes that the mouse is currently in.

\subsubsection{\textit{inFigure} (Method)}
\label{sssec:.modeling.occamtools.OccamPointPicker.inFigure}

gets the figure number that the mouse is in

\subsubsection{\textit{on\_close} (Method)}
\label{sssec:.modeling.occamtools.OccamPointPicker.on_close}

close the figure with a 'q' key event and disconnect the event ids
        

\subsection{\textit{Occam2DData} (Class)}
\label{ssec:.modeling.occamtools.Occam2DData}

Occam2DData covers all aspects of dealing with data for an Occam 2D
    inversion using the code of Constable et al. [1987] and deGroot-Hedlin and 
    Constable [1990] from Scripps avaliable at \url{    http://marineemlab.ucsd.edu/Projects/Occam/2DMT/index.html}


\subsubsection{\textit{make2DdataFile} (Method)}
\label{sssec:.modeling.occamtools.Occam2DData.make2DdataFile}

Make a data file that Occam can read.  At the moment the inversion line
        is the best fit line through all the stations used for the inversion.

\subsubsection{\textit{read2DdataFile} (Method)}
\label{sssec:.modeling.occamtools.Occam2DData.read2DdataFile}

read2DdataFile will read in data from a 2D occam data file.  
            Only supports the first 6 data types of occam2D

\subsubsection{\textit{rewrite2DdataFile} (Method)}
\label{sssec:.modeling.occamtools.Occam2DData.rewrite2DdataFile}

rewrite2DDataFile will rewrite an existing data file so you can 
        redefine some of the parameters, such as rotation angle, or errors for 
        the different components or only invert for one mode or add one or add
        tipper or remove tipper.

\subsubsection{\textit{plotMaskPoints} (Method)}
\label{sssec:.modeling.occamtools.Occam2DData.plotMaskPoints}

An interactive plotting tool to mask points an add errorbars

\subsubsection{\textit{maskPoints} (Method)}
\label{sssec:.modeling.occamtools.Occam2DData.maskPoints}

maskPoints will take in points found from plotMaskPoints and rewrite 
        the data file to nameRW.dat.  **Be sure to run plotMaskPoints first**

\subsubsection{\textit{read2DRespFile} (Method)}
\label{sssec:.modeling.occamtools.Occam2DData.read2DRespFile}

read2DRespFile will read in a response file and combine the data with info 
        from the data file.

\subsubsection{\textit{plot2DResponses} (Method)}
\label{sssec:.modeling.occamtools.Occam2DData.plot2DResponses}

plotResponse will plot the responses modeled from winglink against the 
        observed data.

\subsubsection{\textit{plotPseudoSection} (Method)}
\label{sssec:.modeling.occamtools.Occam2DData.plotPseudoSection}

plots a pseudo section of the data and response if input.

\subsubsection{\textit{plotAllResponses} (Method)}
\label{sssec:.modeling.occamtools.Occam2DData.plotAllResponses}


Plot all the responses of occam inversion from data file.  This assumes
        the response curves are in the same folder as the datafile.
    


\subsection{\textit{Occam2DModel} (Class)}
\label{ssec:.modeling.occamtools.Occam2DModel}

This class deals with the model side of Occam inversions, including 
    plotting the model, the L-curve, depth profiles.  It will also be able to 
    build a mesh and regularization grid at some point.  
    
    It inherits Occam2DData and the data can be extracted from the method
    get2DData().  After this call you can use all the methods of Occam2DData,
    such as plotting the model responses and pseudo sections.

\subsubsection{\textit{read2DIter} (Method)}
\label{sssec:.modeling.occamtools.Occam2DModel.read2DIter}

read an iteration file and combine that info from the 
        datafn and return a dictionary of variables.


\subsubsection{\textit{read2DInmodel} (Method)}
\label{sssec:.modeling.occamtools.Occam2DModel.read2DInmodel}

read an INMODEL file for occam 2D

\subsubsection{\textit{read2DMesh} (Method)}
\label{sssec:.modeling.occamtools.Occam2DModel.read2DMesh}

        reads an Occam 2D mesh file
      

\subsubsection{\textit{get2DData} (Method)}
\label{sssec:.modeling.occamtools.Occam2DModel.get2DData}

get data from data file using the inherited :func:'read2DdataFile' 


\subsubsection{\textit{get2DModel} (Method)}
\label{sssec:.modeling.occamtools.Occam2DModel.get2DModel}

create an array based on the FE mesh and fill the 
        values found from the regularization grid.  This way the array can 
        be manipulated as a 2D object and plotted as an image or a mesh.
        
\subsubsection{\textit{plot2DModel} (Method)}
\label{sssec:.modeling.occamtools.Occam2DModel.plot2DModel}

plot the model output by occam in the iteration file.

\subsubsection{\textit{plotL2Curve} (Method)}
\label{sssec:.modeling.occamtools.Occam2DModel.plotL2Curve}

plot the RMS vs iteration number for the given 
        inversion folder and roughness vs iteration number

\subsubsection{\textit{plotDepthModel} (Method)}
\label{sssec:.modeling.occamtools.Occam2DModel.plotDepthModel}

Plots a depth section profile for a given set of stations.

\subsection{\textit{compare2DIter} (Function)}
\label{sssec:.modeling.occamtools.compare2DIter}


take the difference between two iteration and make a 
    difference iter file


%===============================================================================

\section{\textit{winglinktools} }
\label{sec:modeling.winglinktools}


Collection of tools for dealing with output from Windows WingLink software

\subsection{\textit{readOutputFile} (Function)}
\label{ssec:.modeling.winglinktools.readOutputFile}

read an output file from winglink and output data
    in the form of a dictionary.

\subsection{\textit{plotResponses} (Function)}
\label{ssec:.modeling.winglinktools.plotResponses}

plot the responses modeled from winglink against the
    observed data.

\subsection{\textit{readModelFile} (Function)}
\label{ssec:.modeling.winglinktools.readModelFile}

read  the XYZ .txt-file output by Winglink.

(profile directions restricted to NS or EW !!!)

\subsection{\textit{readWLOutFile} (Function)}
\label{ssec:.modeling.winglinktools.readWLOutFile}

read .out file from winglink

\subsection{\textit{readSitesFile} (Function)}
\label{ssec:.modeling.winglinktools.readSitesFile}

read sites file output from winglink

\subsection{\textit{readSitesFile2} (Function)}
\label{ssec:.modeling.winglinktools.readSitesFile2}

read sites file output from winglink - alternative

more consistent output!

\subsection{\textit{getXY} (Function)}
\label{ssec:.modeling.winglinktools.getXY}

get x (e-w) and y (n-s) position of station and put in middle of cell

connection x-East and y-North here !!!! (ToDo change this)

\subsection{\textit{getmeshblockcoordinates} (Function)}
\label{ssec:.modeling.winglinktools.getmeshblockcoordinates}

(slightly different from ModEM mesh block coordinates!!)
return a list of 3 lists, which again contain the X/Y/Z coordinate of a mesh block

    Orientation is X-North, Y-East, Z-Down.
    Horizontal origin is in the center of the mesh,
    Indexing starts at the lower left (SouthWest) corner


%===============================================================================

\section{\textit{ws3dtools} }
\label{sec:modeling.ws3dtools}

Handling Weerachai's Ws3Dinv code


\subsection{\textit{ListPeriods} (Class)}
\label{ssec:modeling.ws3dtools.ListPeriods}

class to pick periods

\subsubsection{\textit{connect} (Method)}
\label{sssec:modeling.ws3dtools.ListPeriods.connect}

?

\subsubsection{\textit{onclick} (Method)}
\label{sssec:modeling.ws3dtools.ListPeriods.onclick}

?

\subsubsection{\textit{disconnect} (Method)}
\label{sssec:modeling.ws3dtools.ListPeriods.disconnect}

?


\subsection{\textit{writeWSDataFile} (Function)}
\label{ssec:modeling.ws3dtools.writeWSDataFile}

writes a data file for WSINV3D from winglink outputs

\subsection{\textit{writeInit3DFile} (Function)}
\label{ssec:modeling.ws3dtools.writeInit3DFile}

Makes an init3d file for WSINV3D

\subsection{\textit{writeStartupFile} (Function)}
\label{ssec:modeling.ws3dtools.writeStartupFile}

makes a startup file for WSINV3D

\subsection{\textit{readDataFile} (Function)}
\label{ssec:modeling.ws3dtools.readDataFile}

read in Ws3Dinv data file


\subsection{\textit{plotDataResPhase} (Function)}
\label{ssec:modeling.ws3dtools.plotDataResPhase}

plot responses from the data file and if there is a response file

\subsection{\textit{plotTensorMaps} (Function)}
\label{ssec:modeling.ws3dtools.plotTensorMaps}

plot phase tensor maps for data and or response, each figure is of a
    different period.  If response is input a third column is added which is 
    the residual phase tensor showing where the model is not fitting the data 
    well.  The data are plotted in km in units of ohm-m.

\subsection{\textit{readModelFile} (Function)}
\label{ssec:modeling.ws3dtools.readModelFile}

read in a Ws3Dinv model file

