\section{\textit{z} }
\label{sec:core.z}

Module for dealing with the MT impedance tensor $\mathbf{Z}$. The main class (Z) inherits a class that handles data files (Edi). Attributes of $\mathbf{Z}$ can be determined, visualised, and converted. 

\subsection{\textit{Edi} (Class)}
\label{ssec:core.z.Edi}

A class for reading EDI files. Additionally, their contents can be changed and the files can be rewritten. 

\subsubsection{\textit{readEDI} (Method)}
\label{sssec:core.z.Edi.readEDI}

Read in Edi files.

\subsubsection{\textit{rewriteedi} (Method)}
\label{sssec:core.z.Edi.rewriteedi}

Write out Edi files.

\subsection{\textit{Z} (Class)}
\label{ssec:core.z.Z}

Defining an impedance tensor object. This can then be analysed, transformed, visualised. For instance, REsistivity-Phase-Plots can be generated. 

\subsubsection{\textit{getInvariants} (Method)}
\label{sssec:core.z.Z.getInvariants}

.

\subsubsection{\textit{getPhaseTensor} (Method)}
\label{sssec:core.z.Z.getPhaseTensor}

.

\subsubsection{\textit{getTipper} (Method)}
\label{sssec:core.z.Z.getTipper}

.

\subsubsection{\textit{removeDistortion} (Method)}
\label{sssec:core.z.Z.removeDistortion}

.

\subsubsection{\textit{removeStaticShift} (Method)}
\label{sssec:core.z.Z.removeStaticShift}

.

\subsubsection{\textit{getResPhase} (Method)}
\label{sssec:core.z.Z.getResPhase}

.

\subsubsection{\textit{getResTensor} (Method)}
\label{sssec:core.z.Z.getResTensor}

.

\subsubsection{\textit{plotResPhase} (Method)}
\label{sssec:core.z.Z.plotResPhase}

.

\subsubsection{\textit{plotPTAll} (Method)}
\label{sssec:core.z.Z.plotPTAll}

.

\subsubsection{\textit{plotTipper} (Method)}
\label{sssec:core.z.Z.plotTipper}

.


\subsection{\textit{PhaseTensor} (Class)}
\label{ssec:core.z.PhaseTensor}

    Generates a  phase tensor object from given $\mathbf{Z}$, following 
    Caldwell et al. [2004].

\subsection{\textit{ResPhase} (Class)}
\label{ssec:core.z.ResPhase}
ResPhase is a resistivity and phase class...

\subsection{\textit{Tipper} (Class)}
\label{ssec:core.z.Tipper}

Generates a Tipper element from tipper vector and given rotation.

\subsection{\textit{Zinvariants} (Class)}
\label{ssec:core.z.Zinvariants}

Class for holding just the invariants of $\mathbf{Z}$

\subsection{\textit{ResistivityTensor} (Class)}
\label{ssec:core.z.ResistivityTensor}

  gets components of the resistivity tensor defined by 
    O'reilly [1978] and Weckmann et al. [2003]

\subsection{\textit{PhaseTensorResidual} (Class)}
\label{ssec:core.z.PhaseTensorResidual}
  
  calculates the tensor residual as defined by 
$ \Delta \Phi_{1,2} := \hat{\mathbf{I}} - \Phi_1^{-1} \Phi_2$

\subsection{\textit{ResistivityTensorResidual} (Class)}
\label{ssec:core.z.ResistivityTensorResidual}

calculate the resistivity residual between two tensors
    $ \Delta \Phi_{1,2} := \hat{\mathbf{I}} - \Phi_1^{-1} \Phi_2$
    ????


%==========================================================

\section{\textit{mttools} }
\label{sec:core.mttools}

Module for helping with all general MT analysis steps


\subsection{\textit{convertlpB} (Function)}
\label{ssec:core.mttools.convertlpB}

Convert the magnetic field from counts to units of $\mu V/nT$.

\subsection{\textit{convertE} (Function)}
\label{ssec:core.mttools.convertE}

Convert the electric field from counts to units of $\mu V/m$.

\subsection{\textit{rotateE} (Function)}
\label{ssec:core.mttools.rotateE}

Rotate the electric field to geographic north

\subsection{\textit{rotateB} (Function)}
\label{ssec:core.mttools.rotateB}

Rotate the magnetic field such that $B_x$ is pointing to
    magnetic north and $B_y$ to geomagnetic east.

\subsection{\textit{padzeros} (Function)}
\label{ssec:core.mttools.padzeros}

return a function that is padded with zeros to the next power of 2

\subsection{\textit{filter} (Function)}
\label{ssec:core.mttools.filter}

apply a sinc filter of width w to the function f by multipling in the frequency domain.

\subsection{\textit{dctrend} (Function)}
\label{ssec:core.mttools.dctrend}

remove a DC trend

\subsection{\textit{normalizeL2} (Function)}
\label{ssec:core.mttools.normalizeL2}

return the function f normalized by the L2 norm:$  \frac{f(x)}{\sqrt(\sum_i |x_i|^2 )}$.

\subsection{\textit{decimatef} (Function)}
\label{ssec:core.mttools.decimatef}

decimate a function by the factor m. First an 8th order Cheybechev 
    type I filter with a cuttoff frequency of $0.8/m$ is applied in both 
    directions to minimize any phase distortion and remove any aliasing.

\subsection{\textit{openMTfile} (Function)}
\label{ssec:core.mttools.openMTfile}

Open an MT raw data file, convert counts to units and return an 1D-array.

\subsection{\textit{convertCounts2Units} (Function)}
\label{ssec:core.mttools.convertCounts2Units}

Conversion counts -> units ...???

\subsection{\textit{combineFewFiles} (Function)}
\label{ssec:core.mttools.combineFewFiles}

Combine files in a directory path (dirpath) that have a given start and 
    end time in the form of (HHMMSS).  It looks for files at cachelength then 
    decimates the data.

\subsection{\textit{combineFiles} (Function)}
\label{ssec:core.mttools.combineFiles}

Combine raw data files from different days into one file.

\subsection{\textit{adaptiveNotchFilter} (Function)}
\label{ssec:core.mttools.adaptiveNotchFilter}

apply a notch filter to an array by finding the nearest peak around
    the supplied notch locations.  The filter is a zero-phase Chebyshev type 1 
    bandstop filter with minimal ripples.

\subsection{\textit{removePeriodicNoise} (Function)}
\label{ssec:core.mttools.removePeriodicNoise}

take average a window of length noise period and 
    average the signal for as many windows that can fit within the data.  This
    averaged window is convolved with a series of delta functions at each window
    location to create a noise time series. This is then subtracted from the 
    data to get a 'noise free' time series.

\subsection{\textit{imp2resphase} (Function)}
\label{ssec:core.mttools.imp2resphase}

convert impedances z[4,3,len(freq)] to resistivities (ohm-m) and phases (deg) as well as the 
    errors of each.  Note the phase is calculated using arctan2 putting the 
    phase in the correct quadrant.  The xy component is placed into the positive
    quadrant by adding 180 deg. 

    Does NOT take impedance objects though!!

\subsection{\textit{sigfigs} (Function)}
\label{ssec:core.mttools.sigfigs}

return a string with the proper 
    amount of significant digits for the input number

\subsection{\textit{writeedi} (Function)}
\label{ssec:core.mttools.writeedi}

write an .edi file for one station

\subsection{\textit{rewriteedi} (Function)}
\label{ssec:core.mttools.rewriteedi}

rewrite an edifile for one station, say if it needs to be rotated 
    or distortion removed.

\subsection{\textit{getnum} (Function)}
\label{ssec:core.mttools.getnum}

get number from string list and put it into an array ??????

\subsection{\textit{readedi} (Function)}
\label{ssec:core.mttools.readedi}

read in an edi file written in a format given by
    \url{http://www.dias.ie/mtnet/docs/ediformat.txt}.
    Returns: lat,lon,frequency,Z[zreal+i*zimag],Zvar,tipper,tippervar (if 
    applicable)

\subsection{\textit{combineEdifiles} (Function)}
\label{ssec:core.mttools.combineEdifiles}

combine edifile1 with
    edifile2 according to nread1 (integer of number of frequencies to read in from edifile1.Index is from the start, ie [0:nread1]) and nread2 (integer of number of frequencies to read in from edifile2. Index is from end of frequencies, ie [-nread2:]). It will combine frequencies, impedance and tipper.
    
    Note nread1 is from the start for edifile1 and nread2 is from end for 
    edifile2.

\subsection{\textit{sil} (Function)}
\label{ssec:core.mttools.sil}

split a single line written in an .ini file
    for burpinterface and return the list of strings.

\subsection{\textit{readStationInfo} (Function)}
\label{ssec:core.mttools.readStationInfo}

read in a .txt (tab 
    delimited) or .csv(comma delimited)file that has the following information: 
    station name, latitude, longitude, elevation, date collected, 
    components measured (a number: 4 for $E_x, E_y, B_x,B_y$, 5 for 
    $E_x, E_y, B_x,B_y, B_z$, 6 for $E_x, E_y, B_x,B_y, B_z, TP$, magnetic type (BB or LP), dipole 
    lengths (m), data logger gain (very low=2.5,low=1,high=.1),interface box 
    gain (10 or 100), $B_z$ correction for longperiod data and the $ B_x,B_y, B_z$ components as they were measured in the field. Use headers: station, lat, lon, elev, date, mcomps, magtype, ex, ey, dlgain, igain, lpbzcor, magori. 
    Returns a list of the dictionaries with found information.

\subsection{\textit{getStationInfo} (Function)}
\label{ssec:core.mttools.getStationInfo}

returns info for the nominated
    station from the stationinfofile as a dictionary with key words in the 
    hdrinfo.

\subsection{\textit{convertfiles} (Function)}
\label{ssec:core.mttools.convertfiles}

convert data of counts from data logger to units....again????

\subsection{\textit{removeStaticShift} (Function)}
\label{ssec:core.mttools.removeStaticShift}

remove static shift by calculating the median of respones of nearby stations, within a given radius.  If the 
    ratio of the station response to the median on either side is out of 1$\pm$tolerance then 
    the impedance tensor for that electric component will be corrected for 
    static shift.

\subsection{\textit{makeDayFoldersFromObservatoryData} (Function)}
\label{ssec:core.mttools.makeDayFoldersFromObservatoryData}

take observatory data and put it
    into day folders for processing.

\subsection{\textit{makeKML} (Function)}
\label{ssec:core.mttools.makeKML}

make a kml file for Google Earth to plot station locations 
    relatively quickly

\subsection{\textit{getPeriods} (Function)}
\label{ssec:core.mttools.getPeriods}

Plots periods for all stations in edipath and the plot is interactive, just
    click on the period you want to select and it will appear in the console,
    it will also be saved to lp.plst.  To sort this list type\\ \texttt{lp.plst.sort()}
    
    The x's mark a conformation that the station contains that period.  So 
    when looking for the best periods to invert for look for a dense line of 
    x's
