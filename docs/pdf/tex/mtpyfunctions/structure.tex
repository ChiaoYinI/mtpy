\section{{overview} }
\label{sec:structure.overview}

MTpy is organised in subpackages. They are separated by the main topic of the contained modules and functions. 

Besides these subpackages, there is an extra folder, which contains available documentation.


\section{{subpackages} }
\label{sec:structure.subpackages}

The existing subpackes are 
\begin{itemize}
\item \textit{core}{~}\\
These modules deal with the elementary steps of general MT processing steps (mttools) and  the attributes of the impedance tensor (Z).
\item \textit{imaging}{~}\\
Modules and functions, which deal with the general visualisation within the analysis of MT data. More specific plotting routines are around in other subpackages (e.g. processing).  
\item \textit{modeling}{~}\\
Wrapper and auxilliary functions for the handling of external standard MT inversion and modelling tools: WingLink, Occam, Ws3Dinv, ModEM,...
\item \textit{processing}{~}\\
Wrapper and functions for the processing of MT data. Mainly concentrating on BIRRP.
\item \textit{utils}{~}\\
Tools and scripts for helping with small or very specific tasks. For instance conversions of units, data formats, or modeltypes. To be included later: GUIs for running BIRRP, OCCAM, Ws3Dinv.


\end{itemize} 
After the installation of MTpy, the respective subpackages can be imported with \\
\texttt{import mtpy.<subpackagename>}.\\ 
For the direct import of a module type\\ 
\texttt{import mtpy.<subpackagename.<modulename>} instead.

\section{{command line scripts} }
\label{sec:structure.commandline}

Command line scripts are directly callable from the terminal (provided Python ios installed). Arguments are following the scriptname, all separated by just a blank space.

From within an IPython session, these scripts can be called as if the call was from the terminal by putting a preceeding \texttt{run} in front of the call. 